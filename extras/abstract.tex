\prefacesection{Abstract}

The focus of this thesis is on understanding the pulsar magnetosphere particularly
with the exploitation of radio polarization.  
The rotating vector model (the standard pulsar polarization
model) can be modified to estimate emission altitude.  However, more
realistic treatment of relativistic and sweep-back effects show that such
augmentations break down relatively quickly. This break-down is quantified and
fitting functions are provided that preserve accuracy to higher altitude.  The
rotating vector model has proven particularly poor for modeling energetic
pulsars.  We remedy this by including the effects of orthogonal mode jumps,
multiple emission altitudes, open zone growth via y-point lowering, and
interstellar scattering.  A large number of discrepancies can be
understood while retaining the geometric picture.  The model is systematically applied to six
{\it Fermi}-detected pulsars (PSR J0023$+$0923, PSR J1024$-$0719, PSR J1744$-$1134,
PSR J1057$-$5226, PSR J1420$-$6048, and PSR J2124$-$3358).
Several other examples of utilizing polarization modeling to understand
pulsar geometry and emission characteristics are also presented.
Finally, a
bidirectional emission model is applied to the polarization data of PSR J1057$-$5226
and PSR J1705$-$1906.  
Overall, we push the limit of what
can be learned from a geometrically based model and emphasize the importance of
tying such models to data.  

