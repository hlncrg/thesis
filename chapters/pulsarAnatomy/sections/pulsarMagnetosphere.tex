\section{Pulsar Magnetosphere}
\label{sec:pulsarMag}
The simplest model of the pulsar magnetosphere is
a very large magnetic (static) dipole in space.  Many of 
the well-used analytic formulations for analysis of pulsar
data are based on this simplistic assumption. 
We will discuss some of the models in later chapters.

The magnetic field of a \textit{rotating} vacuum dipole are given as

\begin{equation}
\label{eq:magField}
\vec{B}=-\frac{ \vec{m}(t_{\rm r}) +r\dot{\vec{m}}(t_{\rm r}) + r^2\ddot{\vec{m}}(t_{\rm r})}{r^3} + \left[ \frac{ 3\vec{m}(t_{\rm r}) +3r\dot{\vec{m}}(t_{\rm r}) + r^2\ddot{\vec{m}}(t_{\rm r})}{r^3} \cdot \mathbf{\hat{r}} \right] \hat{r}
\end{equation}
with
$\vec{m}(t_{\rm r})=m\left[\sin{(\alpha)} \cos{(t_{\rm r} \omega)} \hat{x}+ \sin{(\alpha)} \sin{(t_{\rm r} \omega)} \hat{y} +\cos{(\alpha)} \hat{z}\right]$
\citep{kaburaki1980determination}.
Here retarded time is $t_{\rm r}=t-r/c$
and $m$ is dipole magnetic moment.
This magnetic field contains information about relativistic effects
as well as a sweep-back form from the spin of the neutron star, making
it much more realistic than the classical dipole.  Conversely, it is much more difficult
to use in analytical closed-form formulations without approximations.
We use this form of the magnetic field extensively in modeling
of both the radio and $\gamma$-ray emission in our computational 
models.

In the pulsar magnetosphere, two primary regions emit photons: the polar gap
and the outer gap.  The polar gap is located near the magnetic pole of the pulsar
and relatively low in the magnetosphere near the surface of the neutron star.
Here the magnetic field is the strongest.   Emission is produced by curvature 
radiation.  Further, interaction of photons within the strong magnetic field
produce electron-position pair that then cause a cascade effect.  The
polar gap is the classical location of radio emission.

The outer gap is located further out in the magnetosphere in comparison to 
the polar gap and is also in a wider range of altitudes as it is said
to follow particular field lines.  The existence of the outer gap
is based on the Goldreich-Julian density: 
\begin{equation}\rho=\frac{\vec{\nabla}\cdot \vec{E}}{4\pi}=-\frac{\vec{\Omega}\cdot \vec{B}}{2\pi}.\end{equation}
The Goldreich-Julian density is formulated from Gauss' Law applied
to a force free model:
\begin{equation}\vec{F}=\vec{E}+(\vec{\Omega}\times \vec{r})\times \vec{B}/c=0.\end{equation}
This density arises from the spin of the neutron star 
(a spherical electric conductor) in  
its own magnetic field. Such spinning causes a \textit{nonzero}
force but free charges from the neutron star rearrange
according to the Goldreich-Julian density
to cancel this rotation-induced electromotive force.

The outer gap is located on the outer side of the null charge surface,
a cone-like surface originating from the polar cap (and within
the region of open field lines).  
The open field lines are those that extend beyond the pulsar light cylinder
never to return to the other magnetic pole.
The light cylinder radius, $R_{\rm{LC}}$, is the distance from the
center of the neutron star at which co-rotating particles would be traveling at the
speed of light.
The null charge surface
is located at $\vec{B}\cdot\vec{\Omega}=0$ in the magnetosphere.  This surface
is then defined by the location of the curvature of the magnetic 
field lines changing direction in relation to the spin axis but is
also the location where $\rho=0$.  One side of the surface is of one charge
and the other side of the surface is of the opposite charge.  

Further, if the charged particles are located in the region of open 
field lines, the charged particles are not simply locked in the
magnetosphere but can escape along the open field lines out of the 
magnetoshere and into space.  This exodus of charged particles is
favorable near the null charge surface because of the proximity 
of charged particles of opposite sign.  

This outflow creates a gap in the charge density (the outer gap)
where free charges can accelerate to relativistic energies and
emit in the $\gamma$-rays.  Further, this charge-starved
region is the location of cascade processes of pair-production. 

Charged particles in the magnetosphere are argued to co-rotate 
with field lines while traveling along the field lines.
In particular, we can note that gyrating motion of the 
particles around field lines will dissipate quickly through
synchrotron radiation.  
Synchrotron lifetime of electron is $T_{\rm s}=(5.1\times10^8 /B^2)\sqrt{1-v^2/c^2} $ s,
where $B$ is in Gauss
\citep{lyne2006pulsar}.
This timescale is much smaller than
the travel time of a particle following a field line thus
any gyration will be dissipated.

A third gap has been argued to exist that bridges the outer gap region
and the polar cap gap and is called the slot gap.  
This gap is thin and extends from the polar cap to the light cylinder following
the last closed field lines.
The $\gamma$-ray model with emission from
this gap is called the two-pole caustic model \citep{dyks2003two}. 
This model will be mentioned again in Chapter \ref{chapter:collaborationWork} 
in the work of $\gamma$-ray modeling in connection to polarization modeling.
