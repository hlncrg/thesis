\section{Neutron Star and Equation of State}
\label{sec:NSandEOS}
The surface of the neutron star is a rigid crystalline 
surface of iron nuclei.  With increasing depth this
lattice crust is made up of increasingly heavy nuclei.  
For a sufficiently large depth, free electrons 
penetrate the lattice.  At further depths, neutrons
are free in the lattice in the ``neutron drip'' region
\citep{kraus1986radio}.

Beyond this crust is an outer core made of 
neutron super-fluid and proton superconductor.
The material is around $95\%$ neutrons.
During rotation, the crust and the core 
can decoupled due to slippage, causing star quakes.
The quakes are seen as glitches in the pulsar phase data.

The neutron star interior is made up of ultra-dense nuclear material
beyond any known substance obtainable in a laboratory.  Neutron
stars are therefore valuable probes of the fundamental nature of 
matter.  Numerous theoretical equations of state for the stellar
core exist in the literature
although only one can be the correct and 
true equation.  Accurate measurements of neutron stars with 
extreme mass are often used to rule out several of these formulations.
The exact composition of this inner core is still
an area of active research (see e.g. for overviews \citealp{becker2009neutron}).
