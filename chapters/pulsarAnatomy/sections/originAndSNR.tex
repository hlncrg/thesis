\section{The Origin of Pulsars and Supernova Remnants}
\label{sec:originAndSNR}
A star will eventually collapse under its own gravity once it has exhausted its energy source.
If the star is massive enough, such a transition will come about violently in a
supernova explosion where most of the mass is expelled in a shell around the position
the star once was. This glowing mass is known as a supernova remnant which
we can continue to observe for 100 to 1000 years after the initial explosion.  The remainder of the
star collapses further either into a black hole or a neutron star depending on the
original mass of the star.  Since most of the angular momentum is conserved, the
much smaller, denser neutron star has a very small rotational period.
Additionally, magnetic flux is also conserved resulting in large magnetic fields.

For a normal star, the radius $R_0$ is $10^5$ to $10^8$ km and
the rotational period $P_0$ is a month to several years.
Equations for conservation of angular momentum and magnetic flux are
\begin{equation}
M R^2_0 \Omega_0=M R^2 \Omega
\end{equation}
\begin{center}
and
\end{center}
\begin{equation}
R^2_0 B_0 = R^2 B.
\end{equation}
With some physical considerations
(e.g. a very small period would result
in the star falling apart) it follows that
\begin{equation}
P \sim (R/R_0)^2 P_0 \sim .001 - 1 \quad \rm{seconds}\\\\
\end{equation}
\begin{center}
and
\end{center}
\begin{equation}
B \sim (R_0/R)^2 B_0 \sim 10^{10}-10^{12} \quad \rm{Gauss}.
\end{equation}

The rotational period of pulsars range from a couple of milliseconds to
several seconds.  For instance, PSR J$1748-2446$ad has one of the shortest
 periods known at $1.395$ ms \citep{hessels2006radio}.
The magnetic fields of pulsars are around $10^{10}$ to $10^{12}$ Gauss.
Magnetars, neutron stars with extremely strong magnetic fields, are said to have
magnetic fields of up to $10^{15}$ Gauss.  (The magnetic field of the Earth is
around $\sim 0.4$ Gauss and the magnetic field of the sun is around $\sim 1$ Gauss, for reference.)

One of the most massive pulsars is PSR J$1311-3430$ which has a mass
of $2.15$ to $2.7M_{\odot}$ (solar mass) from spectroscopic measurements \citep{romani2012psr}.
But the radius of neutron stars are tiny compared to solar radius at around $1.4\times10^{-5} R_{\odot}=10$km.
(The circumference is approximately the distance from
San Fransisco to San Jose.)

Neutron stars were first predicted as a result of
supernovae by \cite{baade1934super} many years before they
were observed.
Although the first supernovae was observed thousands of years ago,
the first pulsations (in radio) from pulsars were not observed 
(or more accurately, \textit{recognized}) until
relatively recently in 1967 to 1968.
The first correct and complete explanation of pulsars and their
connection to neutron stars followed rapidly by Gold and Pacini
\citep{pacini1967energy, gold1968rotating, pacini1968rotating}.
Today, the number of radio detected pulsars is in the low thousands and the number of energetic $\gamma$-ray
detected pulsars is in the low hundreds.
The number of galactic neutron stars is estimated at $\sim10^9$ although
how many are actually visible from Earth depends on a number of caveats \citep{colpi1998elusiveness}.



