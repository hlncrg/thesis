\section{Polarization Calculation}
\label{sec:numericalPolarization}

Following a particle along a field line is
important for tracking emission from a particular zone
on the cap usually expressed as a function of $w$,
the fractional distance between the magnetic
pole and the edge of the open-zone cap.
It is also important for calculating the
polarization position angle ($\psi$) of a photon emitted
at a given location.
The position angle is the angle between the spin
axis and the acceleration vector projected onto viewing plane.
In order to calculate the position angle of the emission,
the acceleration of the 
particle which emitted the curvature radiation must be calculated.
We must calculate the properties of the charged particle
one time step forward.

After a time step, the new magnetic 
field at the new location is $\vec{B'}$ (calculated from Equation \ref{eq:magField}).
But in the mean time, our reference frame (the pulsar) has rotated by $\Delta \theta$.
$\Omega=\Delta \theta / \Delta t = 1$ so $\Delta \theta = 1/25000$
Because of this rotation, the point is actually located at:

\begin{equation}
\left( \begin{array}{c}
x'' \\
y'' \\
z'' \end{array} \right)
\qquad
 \begin{array}{c}
  \\
= \\
  \end{array}
\qquad
\left( \begin{array}{ccc}
\cos\Delta\theta & -\sin\Delta\theta & 0 \\
\sin\Delta\theta &  \cos\Delta\theta & 0 \\
0                & 0                 & 1 \end{array} \right)
\qquad
\left( \begin{array}{c}
x' \\
y' \\
z' \end{array} \right)
\qquad
 \begin{array}{c}
  \\
= \\
  \end{array}\end{equation}
\begin{equation}
\qquad
\left( \begin{array}{c}
x'\cos\Delta\theta - y'\sin\Delta\theta \\
x'\sin\Delta\theta - y'\cos\Delta\theta \\
z' \end{array} \right). \end{equation}

Also the magnetic field is now

\begin{equation}
\begin{array}{c}
             \\
\vec{B''}= \\
             \end{array}
\qquad
\left( \begin{array}{ccc}
\cos\Delta\theta & -\sin\Delta\theta & 0 \\
\sin\Delta\theta &  \cos\Delta\theta & 0 \\
0                & 0                 & 1 \end{array} \right)
\qquad
\begin{array}{c}
             \\ 
\vec{B'} \\
             \end{array}. \end{equation}

Again, the co-rotation velocity is $\vec{v}_{\rm t}''=<-y'',x'',0>$ and
the magnitude of the velocity from the magnetic field at this new location is
\begin{equation}
|\vec{v}_B''|=-\vec{v}_{\rm t}'' \cdot \hat{v}_B'' + \sqrt{(\vec{v}_B'' \cdot \hat{v}_B'')^2 - v_{\rm t}''^2 +1}
.\end{equation}
As before, we can calculate the total velocity; using this and the previous
velocity, we can calculate the acceleration of a particle. 
The change in the velocity is given by  $\Delta t \vec{a} = \vec{v}_{\rm tot}'' - \vec{v}_{\rm tot}$.

As stated earlier, the position angle is the angle between the spin axis and the acceleration vector projected onto a viewing
plane.
The projection of the acceleration vector onto the 
viewing plane in the direction of the velocity is given as 
the vector rejection of $\vec{a}$ in direction $<v_x,v_y,v_z>$.
The vector rejection is given as:
\begin{equation}\vec{p}_a=\vec{a} - (\vec{a} \cdot <v_x,v_y,v_z>) <v_x,v_y,v_z>.\end{equation}

Similarly, the project of $<0,0,1>$, the spin axis $\vec{\Omega}$, onto viewing plane is given as
\begin{equation}\vec{p}_{001}= <0,0,1> -(<0,0,1> \cdot <v_x,v_y,v_z>) <v_x,v_y,v_z> = <-v_z v_x,-v_z v_y,1-v_z v_z>.\end{equation}

The projection vectors are normalized and the angle between them simply gives the polarization position angle
of that photon:
\begin{equation}\psi=\arccos(||\vec{p}_a|| \cdot ||\vec{p}_{001}||).\end{equation}
Each position angle is a function of $\alpha$, $\zeta$, and $\phi$.  
If $\alpha$ and $\zeta$ are constant and $\psi$ is plotted versus $\phi$,
the result is the polarization sweep versus the pulsar phase.
