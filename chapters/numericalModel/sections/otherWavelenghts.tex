\section{Numerical (Geometrically-Based) Modeling in Other Wavelengths}
\label{sec:ModelingOtherWavelengths}
Often in studies of pulsar polarization, data in other wavelengths is available
and can be used to cross-compare results obtained from modeling
of the position angle sweep.
In particular, X-ray data of the pulsar wind nebula can 
be used to derive the viewing angle $\zeta$ and
modeling of the $\gamma$-ray light curve derive independent
$\alpha$ and $\zeta$ parameters.  Here we will 
briefly describe these relevant models.

\subsection{Pulsar X-Ray Wind Tori Model} 
\label{subsec:ToriModeling}
The fitting of the pulsar wind tori model in the X-ray band is described in
detail in \cite{ng2004fitting} and this section is but a 
brief overview.  

The X-ray pulsar wind nebula is sometimes visible around
the central pulsar; the orientation of the pulsar and direction of
the spin axis can be 
quantified using the visible nebula.  
The boundary of the pulsar wind nebula is marked by
the location where the ram pressure of the outflowing material equals
the external pressure of the surrounding medium.
The radius of this termination shock boundary is
\begin{equation}
r_{\rm termination}\approx \left(\frac{\dot{E}}{4\pi c P_{\rm ext}} \right)^{1/2}
\end{equation}
where $\dot{E}$ is the spin down energy and $P_{\rm ext}$ is the
external pressure.
\cite{ng2004fitting} assumed simple equatorial
torus with a Gaussian intensity cross section profile
to describe the brightness of the nebula tori.
The fit parameters for this model are $\Psi$, the polar
axis at a position angle; $\zeta$, the viewing angle measured
from the rotation axis; $r$, the radial distance
to the brightest section of the circular tori; $\delta$,
a measure of the width of the circular tori; and $\beta$,
the bulk velocity of the post-shock flow.
Additionally, background must also be fit to 
the image and the point spread function needs to
be applied.

The apparent intensity is given as 
\begin{equation}
I \propto (1-\hat{n}\cdot \beta)^{-(1+\Gamma)} I_0
\end{equation}
\citep{pelling1987scanning}
where $\Gamma$ is the photon spectral index,
$I_0$ is the Synchrotron emission intensity and
$\hat{n}$ is the unit vector along the line of sight.
In terms of usable parameters:
\begin{equation}
I(x^\prime,y^\prime,z^\prime)= 
\frac{N}{(2\pi\delta)^2r} \times
\left(1 - \frac{y^\prime \sin \zeta}{\sqrt{{x^\prime}^2
+{y^\prime}^2}}\beta\right)^{-(1+\Gamma)}
e^{-\left[{z^\prime}^2+(\sqrt{{x^\prime}^2+{y^\prime}^2}-r)^2\right]/2\delta^2}
.\end{equation}
The parameters $x$ and $y$ are the coordinates of the CCD frame and $z$
is the line of sight.  The parameters $x^\prime$, $y^\prime$ and $z^\prime$
are the coordinates in the frame of the tori such that:
\begin{equation}
\begin{array}{ccc}
x^\prime&=&-x\cos \Psi -y\sin \Psi, \\
y^\prime&=&(x\sin \Psi - y\cos \Psi)\cos \zeta + z\sin \zeta, \\
z^\prime&=&-(x\sin \Psi - y\cos \Psi)\sin \zeta + z\cos \zeta.
\end{array}
\end{equation}
The photon spectral index is $\Gamma=1.5$.
For more details and caveats of the model and
fitting scheme, see \cite{ng2004fitting}. 

\subsection{Pulsar $\gamma$-Ray Light Curve Model}

In Section \ref{sec:calculation} we discussed the formulation
necessary to fire photons along magnetic field lines
of a rotating pulsar.  For modeling in the radio,
emission is assumed to come from a single set of magnetic
field lines at the same altitude. For modeling in the 
$\gamma$-rays, emission comes from the outer gap region
of the magnetosphere as discussed in Section \ref{sec:pulsarMag}.
Because emission comes from multiple altitudes, caustics, 
the enveloping of light-rays from multiple locations onto a 
single location, occur.  These caustics give rise to 
sharp peaks in intensity of the $\gamma$-rays. 

Similar to \cite{romani2010constraining}, when modeling
$\gamma$-ray light curves, we apply a path-length cut-off
to the magnetic field lines that are allowed to emit.
Only for $s<2s_{\rm{NC,min}}+\rho_{\rm{ypt}}$ measured 
from the center of the pulsar 
along the arching field line are photons illuminated at
full strength.  Here $s$ is the path length of the 
magnetic field line, $s_{\rm{NC,min}}$ is the minimum
path length of field lines at a given fraction of the polar cap,
and $\rho_{\rm ypt}$ is the effective light cylinder radius.
Beyond which, illumination is damped according
to
\begin{equation}
e^{-[(s-2s_{\rm{NC,min}}-\rho_{\rm{ypt}})/\sigma_s]^2}. 
\end{equation}

This cut-off eliminates field lines that arch over the
pulsar and cross the null-charge surface far from the 
star.  These field lines create a secondary gap disjointed
from the main gap on the other side of the spin axis that
arguably does not exist in most pulsars from a phenomenological
stand point; an exception to this argument is the 
emission from the Crab pulsar \citep{moffett1996multifrequency}.

We must also prescribe radiating zones within the outer gap
and intensity weightings dependent on the location of the radiating
zones.  Typically, only field line at a certain $w$, the fractional distance
from the cap boundary, are considered to emit.  
The heuristic efficiency law 
\begin{equation}
w_0=\sqrt{10^{-33}\rm{erg s^{-1}}/\dot{E}}
\end{equation}
provides an estimation for the choice of $w$.
Theoretical basis for this model is provided in \cite{arons2006theory}
and observational evidence for this model is
provided in \cite{psrcat}.
A Gaussian spread is applied to the sheet defined
by $w_0$ in the magnetosphere to soften the
caustics seen in emission from a single $w$.

The figure of merit $\chi_3$ defined in
\cite{romani2010constraining} is often used for analyzing $\gamma$-ray light
curve data.  This minimization formula is given as
\begin{equation}
\chi_3\propto\sum_i\frac{(M_i-O_i)^2}{O_i} e^{-i/3}
\end{equation}
where $i$ is an index with the value $|M_i-O_i|$ sorted from largest
to smallest.  Here $M_i$ is the model data point
and $O_i$ is the observed data point.  Such a scheme will 
cause models that better match the intensity peaks in the light
curve to be favored during minimization of the function.
  
