\section{Magnetic Field Line Calculation}
\label{sec:calculation}
In this section we discuss how to calculate $\zeta$, the viewing angle, and $\phi$, the pulsar phase
from a photon at a given point in the magnetosphere of a pulsar
with $\alpha$, the angle between rotation axis and magnetic axis.
First we describe the magnetosphere in mathematical terms.
Next we calculate the velocity of a charged particle within
the magnetosphere which dictates the motion of the particle.  
Finally, we describe the projection of photons produced by curvature
radiation from the charged particles onto the field of view.
These photons are the observables from earth.

\subsection{A Point in the Magnetosphere}

In the frame where the magnetic axis is along the $z$ axis, $\theta_{B}(\phi_{B})$ defines
the polar cap edge. The polar cap is defined by the last closed field lines
on the surface of the pulsar.  Field lines originating from within the cap are open
and extend beyond the light cylinder never to return to the surface
of the pulsar.  The light cylinder is the region where co-rotating
particles are traveling at the speed of light.
The fractional distance from the cap edge is given by the parameter $w$,
where $w=1$ is the center of the cap at the magnetic pole and $w=0$ is the edge of the cap.
The angle from the magnetic axis is
$\theta_w=\theta_{B}(1-w)$.  A field line traced from the neutron 
star surface can be defined by a point using
the cap parameters:

\begin{equation}
\begin{array}{ccc}
x= & \sin(\theta_w) \cos(\phi_B), \\
y= & \sin(\theta_w) \sin(\phi_B), \\
z= & \cos(\theta_w). \end{array}
\end{equation}

This point is
in the frame where the magnetic axis is along $\hat{z}$.
This point in the frame where the axis
of rotation is along $\hat{z}$
is obtained by applying a rotation matrix as follows:

\begin{equation}
\begin{array}{c}
             \\
R_{y}(\alpha)\\
              \end{array} 
\qquad
\left( \begin{array}{c}
x  \\
y  \\
z  \end{array} \right)
\qquad
\begin{array}{c}
   \\
=  \\
   \end{array}
\qquad
\left[ \begin{array}{ccc}
\cos \alpha  & 0 & \sin \alpha  \\
0            & 1 & 0            \\
-\sin \alpha & 0 & \cos \alpha  \end{array} \right]
\qquad
\left( \begin{array}{c}
\sin \theta_w \cos \phi_B  \\
\sin \theta_w \sin \phi_B  \\
\cos \theta_w  \end{array} \right)
.\end{equation}

In the new coordinates the starting point of a given magnetic field line is given by:

\begin{equation}
\begin{array}{cc}
x= &  \cos\phi_B \sin\theta_w \cos\alpha + \cos\theta_w \sin\alpha, \\
y= &  \sin\phi_B \sin\theta_w, \\
z= & -\cos\phi_B \sin\theta_w \sin\alpha +  \cos\theta_w \cos\alpha. 
\end{array}
\end{equation}

This can be translated into the more useful spherical coordinates as follows:

\begin{equation}
\begin{array}{cc}
r=      & R_{\rm{NS}}, \\
\theta= & \arctan(\sqrt{(x^2+y^2)/z_w)}\textrm{ }), \\
\phi=   & \arctan(y/x). \\ 
\end{array} 
\end{equation}


\subsection{Velocity and Motion of a Charged Particle}

The velocity ($\vec{v}_{\rm tot}$)
of a charged particle within the magnetosphere 
can be calculated in order
to obtain the direction of travel.
We can calculate the magnetic field strength at this point $(r,\theta,\phi)$ using
Equation \ref{eq:magField}.  The field at this location points in the direction
$\hat{v}_B$.  Charged particles will travel in this direction along the magnetic
field. Additionally, the velocity due to co-rotation with the star is
given by $\vec{v}_{\rm t}=<-y,x,0>$.

We need the \textit{magnitude} of $\vec{v}_B$ to find
$\vec{v}_{\rm tot}=\vec{v}_B+\vec{v}_{\rm t}$;
only the direction is obtained by Equation \ref{eq:magField}.
We estimate $|\vec{v}_B+\vec{v}_{\rm t}|=c=1$ and manipulate
this to

\begin{equation}
\vec{v}_B^2+ 2 \vec{v}_B \cdot \vec{v_{\rm t}} + \vec{v_{\rm t}}^2 = 1.
\end{equation}
By simply expanding out, the equation becomes
\begin{equation}
v_B^2+(\vec{v}_{\rm t} \cdot \hat{v}_B)^2 + 2 \sqrt{v_B^2} \hat{v}_B \cdot \vec{v}_{\rm t} -(\vec{v}_{\rm t} \cdot \hat{v}_B)^2 + v_{\rm t}^2 = 1.
\end{equation}
The above equation can be further manipulated to
\begin{equation}
|\vec{v}_B|=-\vec{v}_{\rm t} \cdot \hat{v}_B + \sqrt{(\vec{v}_{\rm t} \cdot \hat{v}_B)^2 - v_{\rm t}^2 +1}
\end{equation}
thus solving for the magnitude of the velocity on a charged particle from the magnetic field
and $\vec{v}_{\rm tot}=\vec{v}_B+\vec{v}_{\rm t}$ is obtained.

Further, a charged particle moving along the field line will be located
after one time step at approximately:
\begin{equation}
  \begin{array}{c}
x'=x+|v_B|\frac{B_x}{\sqrt{B_x^2+B_y^2+B_z^2}} \Delta t, \\
y'=y+|v_B|\frac{B_y}{\sqrt{B_x^2+B_y^2+B_z^2}} \Delta t, \\
z'=z+|v_B|\frac{B_z}{\sqrt{B_x^2+B_y^2+B_z^2}} \Delta t .\end{array}
\end{equation}
ignoring higher order terms.
We set $\Delta t =1/2500$ for all calculations.

\subsection{Projection onto the $\zeta$--$\phi$ Viewing Plane}

The total velocity can be expressed in spherical coordinates, $(\phi_{\rm v},\theta_{\rm v})$.
This velocity of a charged particle in the magnetosphere is
in the same direction as the direction of a photon emitted tangent to the field line
at the given point.
In order to project this photon onto the plane of the sky, 
the viewing angle is simply given as $\zeta=\theta_{\rm v}$.

The phase at which the photon arrives at an observer is given as
$\phi=\phi_{\rm v} +\Delta\phi_{\rm{travel-time}}$.  The extra piece $\Delta\phi_{\rm{travel-time}}$
is needed because photons emitted at different locations in the magnetosphere
will arrive at the observer at different times even if they are aimed in the same direction.
This change in phase is known as the travel time correction.
The phase difference is related to the speed of light and the spin of the pulsar:
$\Delta\phi_{\rm{travel-time}}=\Omega \Delta t_{\rm{travel-time}}=\Omega \Delta d_{\rm{travel-time}}/c$.
The variable $\Delta d_{\rm{travel-time}}$ is then the distance along the line of sight 
to the observer which is $\Delta d_{\rm{travel-time}}=\vec{v}_{\rm tot} \cdot <x,y,z>$.  
Since $\Omega=1$ and $c=1$, the phase is simply given as
$\phi=\phi_{\rm v} + \vec{v}_{\rm tot} \cdot <x,y,z>$.


