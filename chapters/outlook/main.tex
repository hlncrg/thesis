\chapter{Outlook and Perspective}
\label{chapter:outlook}

\section{Next Steps}
\label{sec:future}

In this thesis, we brought a fresh perspective 
on the modeling of polarization of pulsars
and questioned the current modeling assumptions.
Although the work presented here 
is an improvement on our understanding
of polarization, the need to go still further
is continuously present.
To understand all the data, the model
needs a fundamental change.  Although we did a fair
job in explaining the polarization presented
here, there is still a wealth of pulsar
polarization data that has not been analyzed
and will likely not be understood with the
model revisions presented here alone.

The data that was available for analysis
was limited.  Often times, only a single frequency 
was available for analysis or the frequency with the
most data or smallest error bars was analyzed.
The frequency maps to the altitude \citep{cordes1978observational} 
such that different frequency originate at different
altitudes.
But this mapping    
was never tested due to inadequate data.
As this is a multi-altitude model, having not applied
this model to multi-frequency data within radio
is a blatant shortcoming of this thesis.
Further, the data was averaged over many periods,
averaging out potentially interesting features.  
For instance the polarization of PSR J1057$-$5336
and PSR J1705$-$1906 have periodic nulling as discussed 
in Chapter \ref{chapter:inwardPhotons} which is related to bidirectional
emission; it is possible that analyzing the polarization
data at different times in this cycle may 
yield further insights.
Arguably, a population study is needed
to fully understand the nature of the polarization
data from pulsars instead of focusing so much on
individual sweeps.

Throughout the analysis,
a single altitude of emission for each component
of the polarization was used. 
The use of a single-altitude model cannot produce the
sharp intensity peaks seen in the radio light curve.
In the $\gamma$-ray emission, these sharp peaks in the intensity 
are called caustic peaks and are caused by 
overlapping of emission from a range of altitudes.
In the outer gap model of $\gamma$-ray emission,
these caustic peaks naturally arise from tracing
emission from a single field line transversing through the 
magnetosphere, away from the pulsar.
In the radio model, a single altitude assumption will not 
result in sharp peaks.  A multiple-altitude model would not
be hard to implement with emission for a single component
originating over many altitudes but the prescription of
such an assumption is unclear, thus we assumed the simplest 
model of just a single altitude.

Scattering effects sometimes appear in the polarization
for pulsars where interstellar scattering is low.
For instance, in the paper \cite{karastergiou2009complex},
the author argues a combination of mode
jumps and interstellar scattering
can produce the complex polarization sweeps seen in 
the pulsar data.
The pulsar polarization sweep that the author uses as the
example to which this model is applicable is the data of
PSR B0355$+$54 (PSR J0358$+$5413).
In reality the scattering time constant for 
this pulsar is small compared to the
amount of scattering needed to 
cause the amount of smoothing seen in
the polarization sweep and particularly
between the supposed mode jumps.
We also saw more scattering in PSR J1420$-$6048
than could be explained by the given scattering
time constant (Section \ref{sec:J1420}).
Although we did not formally report the
fitting results of PSR J1600$-$3050 \citep{yan2011polarization},
in preliminary fitting, increasing the scattering time
constant did improve the $\chi^2$ significantly. But
compared to the time constant needed \citep{cordes2002ne2001} 
such fits were unphysical considering the interstellar scattering.
This could indicate internal scattering in the magnetosphere
\citep{braje2001magnetospheric} although such claims would require further investigation
and another set of revisions to the current model.

Our model is based on the vacuum dipole.  This model
is acceptable at low altitude but in some cases,
high altitude limits in the fitting of the
model to the data were explored.  At high altitudes, plasma effects 
become important and a force-free model would be
more appropriate.  

\cite{spitkovsky2006time} first used a time-dependent
numerical code to calculate non-axisymmetric
magnetic fields of a force-free model.
The plasma filled, force-free models that require magnetohydrodynamic
simulations are used for modeling light curves of
high altitude emission in the $\gamma$-rays
\citep{contopoulos2010pulsar,bai2010modeling}.

The time-dependent Maxwell equations are given as
\begin{equation}
\frac{\partial \vec{B}}{\partial t}=-\vec{\nabla} \times \vec{E} 
\end{equation}
\begin{center}
and
\end{center}
\begin{equation}
\frac{\partial \vec{E}}{\partial t}=-\vec{\nabla} \times \vec{B} - 4\pi \vec{J}. 
\end{equation}
And with the force free assumption, we get
\begin{equation}
\vec{J}=\frac{1}{4\pi B^2} \lbrack (\vec{\nabla}\cdot\vec{E}) \vec{E}\times\vec{B} + (\vec{B}\cdot\vec{\nabla}\times\vec{B}-\vec{E}\cdot\vec{\nabla}\times\vec{E})\vec{B}  \rbrack
.
\end{equation}
In the simulations, the above equations are integrated forward in time
\citep{yee1966numerical}
until a steady state is obtained.

\cite{harding2011gamma} modeled $\gamma$-ray light curves
using both vacuum and force-free simulations and
compared to data.
They found that data actually favored
the vacuum models.
Pulsar $\gamma$-ray emission in these models originate
from outer gaps in the plasma-filled magnetosphere
but this study suggests that field lines resemble more
the vacuum models, hinting that the true
solution needs properties of both.

Recent work by \cite{kalapotharakos2012gamma,kalapotharakos2012toward,kalapotharakos2014gamma}
has explored models that
bridge between vacuum and force-free magnetospheres using a finite
conductivity.  
One way of obtaining such a model is by expressing $\vec{J}$
as 
\begin{equation}
\vec{J}=\frac{\vec{\nabla}\cdot\vec{E}}{4\pi} \frac{\vec{E}\times\vec{B}}{B^2} + \sigma \vec{E_{||}}.
\end{equation}
The conductivity $\sigma$ can be varied from 0 to $\infty$
to range between the two standard magnetosphere approximations.
The focus of these papers was again on $\gamma$-ray light curves.
For instance, \cite{kalapotharakos2012gamma} used this
model to synthesize a population study of $\gamma$-ray
pulsars.
But such a model is appropriate for high-altitude radio emission.
Producing polarization sweeps from such a model is sure to 
bring new insights to the magnetospheric emission far from the neutron star.


\section{Summary of Conclusions}
\label{sec:finalConclusion}

The current state-of-the-art, well-used analysis
tool for understanding pulsar polarization is 
the RVM formulation.  Using the simplest model
for analysis is not intrinsically wrong.  Indeed,
when faced with a choice, the simplest model and assumptions
often are the most likely to quickly and clearly
give insights into a physical problem.
Yet, in this thesis we have shown again and again 
the need for a better model to extract all of the
information from the magnetic field lines of pulsars.
In the end, the simple RVM does not work for
complex polarization and for polarization from emission
high in the magnetosphere.  This polarization contains
helpful information about the magnetic field lines which
in turn tells us about the orientation of the pulsar
and the region of emission.  To extract this information,
model assumptions must be revisited.

The limits of 
the analytical models of pulsar polarization including
RVM and beyond were explored.
Through comparison to numerical models, we conclude
that the typical models are
accurate for low altitude emission but
approximations quickly break down at higher altitudes.

Next, in our analysis of complex polarization sweeps from 
high energy pulsars, we included relativistic and
sweep-back effects \citep{romani2010constraining}, 
interstellar scattering \citep{cronyn1970analysis}, 
and orthogonal mode jumps \citep{backer1976orthogonal}. But more
importantly, finite altitudes and multiple altitudes were used to explain a number
of jumps in the polarization data.
The possibility of bidirectional emission from a couple of 
promising pulsars was explored. Overall, there is evidence of high altitude
emission in the radio from the practical application
of the model to the available data.

This thesis work is significant because there are no widely-accepted
alternatives to simple analytical modeling of pulsar polarization position
angles. The analytical model does not
account for effects of the pulsar rotational motion. The simple analytic models
work well for a large number of the known pulsars but fails for young pulsars
and millisecond pulsars which often have complicated polarization. By making
physically motivated models guided by the data, we have been working to
increase the understanding of pulsars emission mechanism and geometry and to
explain discrepancies between data and previous models.


This thesis has laid the groundwork for future endeavors
in extracting knowledge from pulsar polarization data.  It is not an end in
itself but a step beyond the well-established but insufficient
polarization models. Significant steps are still needed
to fully understand the polarization but we have
pushed the field in the right direction to further
this pursuit.


