\chapter{Roadmap}


The main theme of this thesis is the study of 
pulsar polarization.  More in depth, we are interested
in modeling complex polarization of pulsars which
thus far has not been well explained by current models.
This model is strongly tied to available data.
In general, we cover pulsar physics in the
broader context related to the polarization.
We will begin by giving an overview of the 
chapters.

Chapter \ref{chapter:PulsarAnatomy} covers 
high-level information about pulsars.  This is 
quick overview of our understanding of pulsars.
In Section \ref{sec:originAndSNR},
we review the origin of pulsars through supernovae 
explosions and consider the resulting pulsar spin
and pulsar magnetic fields.  We also naturally talk about
the resulting supernova remnant which can 
sometimes be seen around newer pulsars.
In Section \ref{sec:PWN}, we describe some
general properties of the pulsar wind nebula
which will occasionally be visible around pulsars.
We discuss the pulsar magnetospheres in 
Section \ref{sec:pulsarMag}.  This section is particularly
relevant because most of the modeling work later in this
thesis is magnetospheric.
We consider briefly
the neutron star interior in Section \ref{sec:NSandEOS}.
We review in Section \ref{sec:popAndAge}
the range of populations of pulsars and properties of pulsars
based on age.

In Chapter \ref{chapter:radioAndPulsars}, we detail
classical modeling of radio pulsars.  Section
\ref{sec:polarization} covers the basics of polarization
and Stokes parameters in general context.
Next we define interstellar scattering as it relates 
to polarization in Section \ref{sec:interstellarScattering}.  
Interstellar scattering is important in 
several contexts and will be examined again
in the application of models to data.
Sections \ref{sec:beamingGeometry} and \ref{sec:RVMformula}
derive the two standard formulas for radio modeling.
Section \ref{sec:beamingGeometry} obtains
the width of the pulsar conical beam related to the 
phase width of the radio pulse and 
Section \ref{sec:RVMformula} obtains the 
rotating vector formula for polarization position
angle versus phase.


Having discussed the analytically based models,
Chapter \ref{chapter:numericalModels} then describes
how to obtain the numerically implemented models
which we heavily rely upon in this thesis.
Section \ref{sec:calculation} outlines the 
underlying procedure
for calculating the field line structure
of the pulsar.  Section \ref{sec:numericalPolarization} 
elucidates the derivation of the polarization 
from the previous calculation.
Section \ref{sec:ModelingOtherWavelengths} 
describes the numerically
calculated X-ray torus and $\gamma$-ray light
curve which are often used in conjunction
with the polarization model fitting.

In Chapter \ref{chapter:phenomenologicalFitting}
we formally report the shortcomings of using 
the analytical finite-altitude polarization model with 
corrections for the neutron star motion. 
Section \ref{sec:correctionRVM}
offers correction formulas derived from 
the numerical calculations for better 
estimation of the emission height via the
shift between the peak light curve intensity  
and the maximum curvature in the polarization 
sweep.  Sections \ref{sec:heightFromPsi}
and \ref{sec:heightFromWidth} compare 
two other analytical methods for estimating 
emission altitude to numerical results.

Chapter \ref{chapter:tacklingPolarization}
essays the difficulty of modeling energetic
pulsar radio polarization position angle data.
Our model ingredients and fitting methodology
are defined and justified
in Sections \ref{sec:adding} and \ref{sec:fit}.
In Section \ref{sec:app} we apply the
model to a number of pulsars, each highlighting
a different set of aspects of the model.

In Chapter \ref{chapter:collaborationWork}, we 
illustrate the use of polarization modeling
in collaboration with other studies.  The
individual pulsars are analyzed using the
rotating vector model and finite numerical modeling
although overall, we are studying much simpler
data than that seen in Chapter 
\ref{chapter:tacklingPolarization}.  These applications
illustrate the use of polarization in the context of
multi-wavelength studies.  

We were unable to fully understand the 
polarization  of the pulsar PSR J1057$-$5226
in Chapter \ref{chapter:phenomenologicalFitting}.
In Chapter \ref{chapter:inwardPhotons}, we
revisit the data of PSR J1057$-$5226
with an amended model which includes inward-directed
photon emission.  We also examine the 
$\gamma$-ray emission in connection to this model.
We also analyze PSR J1705$-$1906, another pulsar which has 
been suggested to have inward-directed photon
emission from other studies.

In Chapter \ref{chapter:outlook}, we summarize
conclusions from previous chapters in Section \ref{sec:finalConclusion}.
We also discuss in Section \ref{sec:future} 
further improvements and considerations that can be included in 
future work as well and past shortcomings.
