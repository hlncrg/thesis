\section{Effects of Interstellar Medium} \label{sec:interstellarScattering}
The space between us and pulsars is not empty but is full
of ionized medium that effects electromagnetic waves.
Such medium modifies the signal in several ways which we can
quantify.

The dispersion measure is the total density of free electrons ($N(l)$) over the intervening space that the signal from the pulsar
travels:
\begin{equation} DM=\int_0^L N(l) \mathrm{d}l. \end{equation}

The dispersion measure is related to the frequency of the emission ($\nu$) and the time delay of the emission ($t$):

\begin{equation}DM_{[\textrm{cm}^{-3}\textrm{pc}]}=2.410\times10^{-4} \textrm{ } t_{[\textrm{s}]}\textrm{ } \nu^2_{[\textrm{MHz}]}.\end{equation}

This formula is useful for calculating the expected time delay between frequencies:

\begin{equation}
\Delta t_{[\textrm{s}]}= \frac{DM_{[\textrm{cm}^{-3}\textrm{pc}]}}{2.410\times 10^{-4}} \left\{  \frac{1}{\nu^2_{2 [\textrm{MHz}]}} - \frac{1}{\nu^2_{1 \lbrack\textrm{MHz}\rbrack}}  \right\}.\end{equation}

In particular, we wished to quantify the time delay in PSR J1420$-$6048 because the fit was simultaneous
in two frequencies, 10cm and 20cm (Section \ref{sec:J1420}).  
In the end, we found that any time delay due to dispersion
was overpowered by our uncertainty in phase. (See \cite{rohlfs2000tools} for a particularly well formulated
presentation of the dispersion measurement calculation.)

Another effect of interstellar medium on measurements is interstellar scattering.
The scattering of emission in the medium results in multipath propagation.  
The effects of interstellar scattering manifest themselves in pulsar data as
a broadening of the trailing end of the intensity pulse and a flattening of the polarization data versus
phase.

This scattering effect is mathematically described as a convolution of the unscattered
Stokes parameters with a scattering kernel:

\begin{equation}
\begin{array}{l}
I^{\rm scat}=\int I(\phi(t')) g(t-t') dt,\\
Q^{\rm scat}=\int Q(\phi(t')) g(t-t') dt,\\
U^{\rm scat}=\int U(\phi(t')) g(t-t') dt,\\
V^{\rm scat}=\int V(\phi(t')) g(t-t') dt.
\end{array}
\end{equation}
This relation is used for model $Q$ and $U$
in Chapter \ref{chapter:tacklingPolarization}.

Analytical scattering kernels exist assuming various distributions
of interstellar medium.  
The scattering kernel assuming a thin scattering screen 
halfway between source and observer is given by 

\begin{equation}\label{eq:gts2}
g_{ts}(t-t')=\left\{
\begin{array}{lr}
0,                       & t-t' < 0 \\
e^{-(t-t') / \tau_{\rm s}},  & t-t' > 0
\end{array}
\right.
\end{equation}
\citep{williamson1972pulse,williamson1973pulse}.
The scattering kernel assuming a thick screen near the
source is given by 
\begin{equation}
g_{\rm ths}(t-t')=\left\{
\begin{array}{lr}
0,                       & t-t' < 0 \\
\sqrt{\frac{\pi \tau_{\rm s}}{4t^3}} e^{-\pi^2\tau_{\rm s}/16t},  & t-t' > 0.
\end{array}
\right.
\end{equation}
The scattering kernel assuming uniform medium is given by
\begin{equation}
g_{\rm um}(t-t')=\left\{
\begin{array}{lr}
0,                       & t-t' < 0 \\
\sqrt{\frac{\pi^5 \tau_{\rm s}^3}{8t^5}} e^{-\pi^2\tau_{\rm s}/4t},  & t-t' > 0.
\end{array}
\right.
\end{equation}
The variable $\tau_{\rm s}$ is a characteristic scattering time.
The simplest kernel $g_{\rm ts}$ is used in Chapter \ref{chapter:tacklingPolarization}  in 
modeling scattering.

Finally, pulsar polarization data is Faraday rotated
by interstellar medium.  Faraday rotation is a powerful
tool for probing interstellar scattering but for us it is 
a nuisance to be removed from the data.  The degree of Faraday rotation
is related to frequency and can be removed by comparing data
from multiple frequencies.  Considering that $\Delta\psi$ is often
treated as a nuisance parameter, the accuracy of this removal
is not very important.  


