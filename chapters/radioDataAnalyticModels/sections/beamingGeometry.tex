\section{Beaming Geometry}
\label{sec:beamingGeometry}
Assuming a beam of emission centered on the 
magnetic axis, the
relationship between pulse width
in phase and the opening angle of the
cone structure where emission originates
can be derived in closed form.
The cone is centered on the magnetic 
axis defined by 

\begin{equation}
\label{equ:m}
\hat{m}=\cos{(\phi)} \sin{(\alpha)} \hat{x} + \sin{(\phi)} \sin{(\alpha)} \hat{y} + \cos{(\alpha)} \hat{z}
\end{equation}
where $\phi$ is the phase of the pulsar
as it rotates about $\hat{\Omega}=\hat{z}$.
The pulse width is $W$ and thus
for a symmetric cone, the 
pulse is seen between $\phi=-W/2$ and $\phi=W/2$
in phase.
The angle between $\hat{m}$ and the
line of sight at the maximums ($\phi=-W/2$ and $\phi=W/2$) will also be the opening
angle.  The line of sight is given by 
\begin{equation}
\hat{n}=\sin{\zeta} \hat{x}+\cos{\zeta} \hat{z}
\end{equation}
and the relation between opening angle and 
pulse width is
\begin{equation}
\label{equ:ndotm}
\hat{n}\cdot\hat{m}=\cos{\Gamma}=\sin{\zeta} \cos{W/2} \sin{\alpha} + \cos{\zeta} \cos{\alpha} 
.\end{equation}
This equation makes no assumptions about the form of the magnetic field lines.
The angle $\alpha$ is between the spin axis and the magnetic axis and the
angle $\zeta$ is between the spin axis and the line of sight.

We can further relate the pulse width to emission height for a
dipole magnetic field.  The last closed field lines
which define the emission cone for a simple dipole
have the form 
\begin{equation}
\label{equ:dipole}
\mathbf{r}=\sin^3\theta \hat{x} + \sin^2\theta \cos{\theta} \hat{z}
\end{equation}
\begin{center}
and
\end{center}
\begin{equation}
R=|\mathbf{r}|=\sin^2\theta
\end{equation}
in polar coordinates where $R$ is the emission 
altitude measured in $R_{\rm LC}$, light cylinder
radius.
In this coordinate system, $\hat{m}=\hat{z}$.

The angle $\Gamma$ between $\hat{n}$ and $\hat{m}$
will be approximately the angle between 
\begin{equation}
\hat{T}=\frac{d\mathbf{r}/d\theta}{|d\mathbf{r}/d\theta|}
\end{equation}
(the curvature of $\mathbf{r}$) and $\hat{m}$.
This is actually a far field approximation 
but is appropriate because 
the observer is very far from the pulsar.

We can then relate the opening
angle to the altitude using Equation \ref{equ:dipole}.
Again using $\hat{m}=\hat{z}$, Equation \ref{equ:dipole} and Equation \ref{equ:m}:
\begin{equation}\hat{T}\cdot\hat{m}=\frac{2-3\sin^2\theta}{\sqrt{4-3\sin^2\theta}}.\end{equation}

Using $R=\sin^2\theta$ and approximating $R$ as small, 
\begin{equation}\hat{T}\cdot\hat{m}=1-9R/8.\end{equation}

Further using $\hat{T}\cdot\hat{m}\approx\cos{\Gamma}\approx 1-\Gamma^2/2$ and Equation \ref{equ:ndotm},
we get:
\begin{equation}
R=\frac{4}{9}\arccos^2\left[\sin{\zeta}\cos{\frac{W}{2}}\sin{\alpha}+\cos{\zeta}\cos{\alpha}\right]
.
\end{equation}

This equation then relates the width of the intensity pulse to the altitude of
emission.
